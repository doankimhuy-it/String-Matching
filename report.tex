\documentclass[a4paper,11pt]{article}
\usepackage[margin=0.5in]{geometry}
\usepackage[utf8]{inputenc}
\usepackage[utf8]{vntex}
\usepackage{blindtext, amsfonts, amsmath, listings, color}

\usepackage[dvipsnames]{xcolor}
\usepackage{tvietlistings}	

\definecolor{dkgreen}{rgb}{0,0.6,0}
\definecolor{gray}{rgb}{0.5,0.5,0.5}
\definecolor{mauve}{rgb}{0.58,0,0.82}
\definecolor{Ao}{rgb}{0.55, 0.71, 0.0}
% Phần biểu diễn
\lstset{frame=tb,
	language=TeX,
	aboveskip=3mm,
	belowskip=3mm,
	showstringspaces=false,
	basicstyle={\small\ttfamily},
	firstnumber=0,
	numbers=left,
	numberstyle=\tiny\color{gray},
	breaklines=true,
	captionpos=t,
	breakatwhitespace=true,
	tabsize=2,
	moredelim=**[is][\color{red}]{@}{@},
	moredelim=**[is][\color{Ao}]{!}{!},
}

\title{Sections and Chapters}
\author{Gubert Farnsworth}
\date{\today}
\begin{document}
	\maketitle
	\part*{STRING MATCHING}
	
	\section{String Matching Problem}
	Tìm kiếm chuỗi hiện diện trong cuộc sống, ví dụ như: tìm tên thầy dạy DSA trong danh sách giảng viên,
	tìm tên trong bảng điểm,.. hoặc trong khoa học, như là tìm liệu cấu trúc ADN của virus này có trong virus khác hay không,
	
	\section{String Matching Algorithms}
	\begin{enumerate}
		\item Brute-force 
		
			Giải thuật Brute-Force, hay còn gọi là vét cạn, 
			là thuật toán đơn giản nhất trong các thuật toán 
			tìm kiếm chuỗi con \textit{pattern} 
			trong chuỗi cha \textit{text}.

			Có thể giải thích đơn giản, giải thuật Brute-Force 
			so sánh lần lượt mỗi chuỗi con \textit{subtext} của \textit{text} 
			có cùng chiều dài với \textit{pattern} với \textit{pattern}, 
			nếu tìm được, trả về kết quả là vị trí được tìm thấy; khi không 
			tìm được kết quả mong muốn, trả về giá trị quy ước là không tìm thấy.

			Trong ví dụ sau, ta sẽ làm rõ cách hoạt động của giải thuật này:\\
			\textit{text} \hspace*{7mm}= \verb|Let them go!| \\
			\textit{pattern} \hspace*{0.1mm} = \verb|them| 
			% L
			\vspace*{4mm}
			\hrule
			\textcolor{red}{L}et them go!\\
			\textcolor{red}{t}hem

			% Le
			\vspace*{2mm}
			\hrule
			L\textcolor{red}{e}t them go!\\
			\textcolor{white}{L}\textcolor{red}{t}hem
			
			% Let
			\vspace*{2mm}
			\hrule
			Le\textcolor{ForestGreen}{t} them go!\\
			\textcolor{white}{Le}\textcolor{ForestGreen}{t}hem
			
			% Let 
			\vspace*{2mm}
			\hrule
			Le\textcolor{ForestGreen}{t}\textcolor{red}{\textvisiblespace}them go!\\
			\textcolor{white}{Le}\textcolor{ForestGreen}{t}\textcolor{red}{h}em

			% Let t
			\vspace*{2mm}
			\hrule
			Let \textcolor{ForestGreen}{t}hem go!\\
			\textcolor{white}{Let }\textcolor{ForestGreen}{t}hem

			% Let th
			\vspace*{2mm}
			\hrule
			Let \textcolor{ForestGreen}{th}em go!\\
			\textcolor{white}{Let }\textcolor{ForestGreen}{th}em
			
			% Let the
			\vspace*{2mm}
			\hrule
			Let \textcolor{ForestGreen}{the}m go!\\
			\textcolor{white}{Let }\textcolor{ForestGreen}{the}m
						
			% Let them
			\vspace*{2mm}
			\hrule
			Let \textcolor{ForestGreen}{them} go!\\
			\textcolor{white}{Let }\textcolor{ForestGreen}{them}
			\vspace*{2mm}
			\hrule

			Ta tìm thấy chuỗi pattern tại vị trí thứ 4!
			
			Từ ví dụ trên, ta thiết kế mã giả cho giải thuật Brute-Force:

			\begin{lstlisting}
				vị trí tìm thấy = -1
				subtext = chuỗi con đầu text có độ dài bằng pattern
				while (chưa tìm thấy hoặc chưa tới cuối text)
					if (từng ký tự của subtext = pattern):
						trả về vị trí
					else:
						dịch chuyển chuỗi con subtext trong text sang phải 1 chữ cái
				Trả về: vị trí tìm thấy
					\end{lstlisting}
			
			Với chuỗi \textit{pattern} có độ dài là M, chuỗi \textit{text} có độ dài là N \\
			Phân tích độ phức tạp của thuật toán trong trường hợp xấu nhất:
			\begin{itemize}
				\item Mỗi lần so sánh với \textit{subtext}, \textit{pattern} phải so sánh nhiều nhất là M lần (trong trường hợp cả M - 1 ký tự đầu đều đúng).
				\item Có tất cả N - M + 1 chuỗi, vậy số chuỗi cần so sánh nhiều nhất là N - M + 1 \textit{subtext} như vậy (trong trường hợp N - M + 2 chuỗi \textit{subext} đầu không trùng với \textit{pattern})
				
				$\to$ Cần M(N - M + 1) lần. Vì duyệt tới cuối mảng nên đây là trường hợp tìm thấy ở cuối mảng, hoặc không tìm thấy\\
				$\to$ Cận trên $O(MN)$ (vì N > N - M + 1). \\
				$\to$ Cấp phát bộ nhớ: 0.
			\end{itemize}

			Phân tích độ phức tạp của thuật toán trong trường hợp tốt nhất:
			\begin{itemize}
				\item Trong trường hợp tốt nhất, có thể thấy \textit{pattern} chính là \textit{subtext} đầu tiên của \textit{text}.
				\item Như vậy, chỉ cần tốn M lần so sánh các ký tự.
				
				$\to$ Cần M lần. \\
				$\to$ Cận trên $O(M)$. \\
				$\to$ Cấp phát bộ nhớ: 0.
			\end{itemize}
			
			Đánh giá:
			\begin{itemize}
				\item Dễ hiểu, thuật toán này chỉ duyệt từ đầu đến cuối, so sánh tuần tự từng chuỗi con với chuỗi cần tìm kiếm.
				\item Không cần bước tiền xử lý (như các thuật toán được trình bày bên dưới).
				\item Độ phức tạp $O(MN)$. Không cần xin thêm bộ nhớ.
			\end{itemize}
	\end{enumerate}
	
	\part*{PROGRAMING}
	\label{Crossword game}(abc)
	
	
	\setcounter{section}{0}
	\section{Introduce}
	\section{Example Test}
	
\end{document}
